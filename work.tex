
{\sl Tenure-Track Assistant Professor} \hfill 2022 -- Present \\ 
Lyda Hill Department of Bioinformatics and Cecil H. and Ida Green Center for Systems Biology. \newline UT Southwestern Medical Center, Dallas, TX, USA. \hfill \\
Advanced fluorescence imaging of tissues, single cell sequencing, and other single-cell approaches have provided incredible insight into the earliest events in the metastatic cascade, including the growth of the primary tumor and invasion of single cells or cell clusters into adjacent tissues. In contrast, the molecular and cellular processes that enable colonization of a remote tissue are poorly understood and data are scarce. The overarching goal of the Dean Lab is to identify the molecular mechanisms that enable cancer cells to colonize and populate a distant tissue. To achieve this, we are developing a series of self-driving microscopes that leverage cutting-edge computer vision routines and adaptively image diverse specimens, and advancing integrated multiplexing imaging approaches using a combination of hydrogel-based tissue clearing and directed evolution of labeling technologies. 
\newline

{\sl Research-Track Assistant Professor, and Director of the Microscopy Innovation Lab} \hfill 2018 -- 2022 \\ 
Department of Cell Biology. \newline UT Southwestern Medical Center, Dallas, TX, USA. \hfill \\
\forceindent The pace of microscopy technology development far outstrips the rate at which it can be commercialized and delivered to the greater biological community.  The Microscopy Innovation Lab at UT Southwestern works to eliminate this delay by developing potentially transformative instrumentation that is tailored to address specific biological questions, and engaging in in close collaborations with life scientists to gain molecular insight into physiological and pathophysiological phenomena. Towards these means, I work one-on-one with diverse biomedical scientists and physicians to develop custom imaging workflows that synergistically combine advances in optical probes, sample preparation, instrument design and operation, and computer vision-based analyses of TB-scale datasets. Imaging systems include 1) two cleared tissue Axially Swept Light-Sheet Microscopes that achieve 300 and 650 nm isotropic resolution, respectively, throughout centimeter-scale optically cleared tissues, 2) a high-speed single-objective oblique plane light-sheet microscope with 200 and 450 nm lateral and axial resolution, respectively, 3) a hyperspectral TIRF system capable of imaging 6 genetically encoded fluorescent probes in living cells, 4) a 2 and 3-photon laser scanning confocal equipped with adaptive optics, extended depth of focusing, and rapid axial scanning for ultradeep intravital imaging, and 5) a mesoSPIM microscope for macro-scale imaging with 5 $\mu$m isotropic resolution. Furthermore, through collaborations with the Danuser and Fiolka labs, a 6) 2-photon micro-environmental selective plane illumination microscope with 300 nm isotropic resolution, 7) a live-cell 1-photon Axially Swept Light-Sheet Microscope with 350 nm isotropic resolution, 7) a Field Synthesis variant of Lattice Light-Sheet Microscopy, 8) and a multi-directional illumination light-sheet microscope for imaging of developing organisms. Research combines biology, biophysics, optics, and computer vision.  
\newline

{\sl Founder} \hfill 2019 -- 2022 \\ 
Discovery Imaging Systems, LLC, Dallas, TX, USA. \hfill \\
https://www.discoveryimagingsystems.com \\
\forceindent Established an LLC to provide customized and world-class imaging solutions to academic research centers.  Business portfolio focused on new methods for imaging chemically transparent biological specimens with the cleared tissue variants of Axially Swept Light-Sheet Microscopy.  Customer purchases third-party components, and Discovery Imaging Systems provides expert assembly on site and annual maintenance thereafter.

{\sl Postdoctoral Fellow, Fiolka and Danuser Laboratories} \hfill 2014 -- 2018 \\ 
Lyda Hill Department of Bioinformatics \\
UT Southwestern Medical Center, Dallas, TX, USA. \hfill \\
\forceindent Development and application of light-sheet microscopy and computer vision technologies to elucidate mechanisms of cell-extracellular matrix adhesions, Rho GTPase signaling, and cytoskeletal dynamics of breast cancer cells disseminating into 3D tissue-like environments. Ruth L. Kirschstein Postdoctoral Fellow.  Research conducted under the guidance of Drs. Reto Fiolka and Gaudenz Danuser.

{\sl Director, BioFrontiers Advanced Imaging Resource} \hfill 2013 -- 2014 \\ 
BioFrontiers Institute \\
University of Colorado, Boulder, CO, USA. \hfill \\
\forceindent Established first campus-wide microscopy facility and expanded user base expanded to over 100 users from eight departments, two universities, and two biotechnology firms. Led a collaborative team in the building and development of novel microscopy and super-resolution instrumentation and techniques; executed independent research; provided facility strategic vision; organized analytical training for users; worked with researchers to develop customized protocols and assays; collaborated with faculty to compete for grants and contracts to support facility research activities and lead infrastructure grants;  interfaced and worked closely with industrial users.  Directly reported to the Chief Scientific Officer, Dr. Leslie Leinwand, and Dr. Tom Cech.

{\sl Graduate Research Assistant, Palmer Laboratory.} \hfill 2007 -- 2013 \\ 
Department of Biochemistry \\
University of Colorado, Boulder, CO, USA. \\
\forceindent Investigated the role of site-specific mutations in dark-state conversion and irreversible photobleaching for red-fluorescent proteins; developed high-throughput microfluidic cell-sorter capable of measuring fluorescence lifetime and the rate of photobleaching on single mammalian cells; directed-evolution of red-fluorescent proteins for improved photostability; collaborated with other lab members on the development and evaluation of optical biosensors.  Research conducted under the guidance of Drs. Amy E. Palmer and Ralph Jimenez. 

 {\sl Undergraduate Research Assistant, Williamson Laboratory.} \hfill 2003 -- 2006 \\ 
Department of Chemistry \\
Willamette University, Salem, OR, USA. \hfill \\
\forceindent Development of semi-automated laser light scattering instrumentation and the measurement of highly accurate and precise binary liquid-liquid phase diagrams.  Research conducted under the guidance of Dr. J. Charles Williamson.