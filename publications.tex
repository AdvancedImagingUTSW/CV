
\begin{etaremune}

\item Benjamin Nanes et al, Keratin isoforms modulate motility signals during wound healing. {it Submitted}.

\item Philippe Roudot et al, u-track 3D: measuring and interrogating intracellular dynamics in three dimensions.  {\it Submitted}.

\item Erik S. Welf et al, Worrying drives cell migration in mechanically unrestrained environments. {\it Submitted}.

\item Tadamoto Isogai, Kevin M. Dean et al, Direct Arp2/3-vinculin binding is essential for cell spreading, but only on compliant substrates and in 3D. {\it Submitted.}  

\item Andrew D. Weems et al, Blebs Promote Cell Survival by Assembling Oncogenic Signaling Hubs. {\it Nature}. 2023 Mar; Online ahead of print.

\item Alec Bancroft et al, Discoidin Domain Receptor 2 regulates aberrant mesenchymal progenitor cell fate and matrix organization. {\it Sci. Advances.} 2022 Dec; (51) eabq6152.

\item Bingying Chen et al, Increasing the field-of-view in oblique plane microscopy via optical tiling. {\it Biomed. Opt. Exp.} 2022 Nov; (13) 5616-5627.

\item Bingying Chen et al, Resolution doubling in light-sheet microscopy via oblique plane structured illumination. {\it Nat. Methods} 2022 Nov; (11) 1419-1426.

\item Kevin M. Dean et al, Isotropic imaging across spatial scales with axially swept light-sheet microscopy. {\it Nat. Protoc.} 2022 Jul; (17) 2025-2053.
%\begin{center} Featured on the cover of {\it Nat. Protoc.}.\end{center}

\item Sue Y. Kim et al, Particle Retracking Algorithm Capable of Quantifying Large, Local Matrix Deformation for Traction Force Microscopy. {\it PLoS ONE.} 2022 Jun; (17) e0268614. 

\item Iliodora V Pop et al, Structure of long-range direct and indirect spinocerebellar pathways as well as local spinal circuits mediating proprioception. {\it Neursci.} 2022 Jan; (42) 581-600.

\item Hao Liu et al, Heterozygous mutation of Vegfr3 decreases renal lymphatics but is dispensable for renal function.{\it J. Am. Soc. Neph.} 2021 Sep; (32) 3099-3113.

\item Bo-Jui Chang et al, Real-time multi-angle projection imaging of biological dynamics. {\it Nat. Methods.} 2021 Jun; (18) 829-834.
%\begin{center} Featured in the NIH Director's Blog \end{center}

\item Sangyoon J. Han et al, Pre-complexation of talin and vinculin without tension is required for efficient nascent adhesion maturation. {\it eLife.} 2021 Mar; (10) e66151. {\it Co-corresponding author.}

\item Erik S. Welf et al, Actin-membrane release initiates cell protrusion.  {\it Dev. Cell}. 2020 Dec; (55) 723-736.

\item  Etai Sapoznik et al, A versatile Oblique Plane Microscope for large-scale and high-resolution imaging of subcellular dynamics.  {\it eLife.} 2020 Nov; (9) e57681. {\it Co-corresponding author.}
%\begin{center} Featured as a Method to Watch in {\it Nat. Methods} \end{center}

\item Kyung-min Lee et al, Proline rich 11 (PRR11) overexpression amplifies PI3K signaling and promotes antiestrogen resistance in breast cancer. {\it Nat. Comm}. 2020 Oct; (11) 1-15.

\item Tonmoy Chakraborty, Bingying Chen et al, Converting Lateral Scanning Into Axial Focusing to Speed Up 3D Microscopy. {\it Light Sci. Appl.} 2020 Sep; (9) 00401-00409.

\item Bo-Jui Chang, Kevin M. Dean, and Reto Fiolka.  Systematic and quantitative comparison of lattice and Gaussian light-sheets.  {\it Opt. Express.} 2020 Aug; (28) 27052-27077.
%\begin{center} Highlighted as an Editor's Pick in {\it Opt. Express} \end{center}

\item Bingying Chen, Tonmoy Chakraborty et al,  Extended depth of focus multiphoton microscopy via incoherent pulse splitting. {\it Biomed. Opt. Express.} 2020 Jun; (11) 3830-3842.

\item Tonmoy Chakraborty et al, Light-sheet microscopy of Cleared Tissues with Isotropic, Subcellular Resolution. {\it Nat. Methods.} 2019 Nov; (16), 1109-1113.  {\it Co-corresponding author.}
%\begin{center} Technology Feature in {\it Nature} \end{center}

\item Meghan K. Driscoll et al, Robust and automated detection of subcellular morphological motifs in 3D microscopy images. {\it Nat. Methods.} 2019 Sept; (16) 1037-1044.  

\item Ashwathi S. Mohan et al, Enhanced dendritic actin network formation in lamellipodia drives proliferative signaling in growth-challenged Rac1\textsuperscript{P29S} melanoma cells.  
{\it Dev. Cell}.  2019 May; (49) 444-460.

\item Bo-Jui Chang, Mark Kittisopikul et al, Universal Light-Sheet Generation with Field Synthesis. {\it Nat. Methods}. 2019  Feb; (16) 235-238. 
%\begin{center}  Featured in {\it Nature Methods} "News \& Views." \end{center}

\item Kevin M. Dean and Reto Fiolka.  Lossless Three-Dimensional Parallelization in Digitally Scanned Light-Sheet Fluorescence Microscopy.  {\it Sci. Rep.} 2017 Aug; (7), 9332.  

\item Kevin M. Dean et al, Imaging Subcellular Dynamics with Fast and Light-Efficient Volumetrically Parallelized Microscopy. {\it Optica}.  2017 Feb; (4), 263-271.
%\begin{center} Featured on the cover of {\it Optica} and {\it Nature Methods} "News and Views." \end{center}

\item Jun Chu et al,  A Bright Cyan-Excitable Orange Fluorescent Protein for Dual-Emission Microscopy and Highly Sensitive Bioluminescence Imaging In Vivo.  {\it Nat. Biotechnol}. 2016 May 30; (34), 760-767.

\item J. Charles Williamson et al.  Determination of Liquid-Liquid Critical Point Composition Using 90$^{\circ}$ Laser Light Scattering. {\it Phys. Rev. E.}  2016 Apr 21; (93), 042610.

\item Kevin M. Dean, Philippe Roudot et al,  Diagonally Scanned Light-Sheet Microscopy for Fast Volumetric Imaging of Adherent Cells.  {\it Biophys. J}.  2016 Mar 29; (110), 1456-1465. 
%$\newline 
%\ddagger$ {\it These authors contributed equally to this work.}
%\begin{center} Featured as an 'Emerging Biophysical Method' and the 'Best Of' reprint collection of {\it Biophysical Journal}. \end{center}

\item Erik S. Welf, Meghan K. Driscoll et al, Quantitative Multiscale Cell Imaging in Controlled 3D Microenvironments. {\it Dev. Cell}.  2016 Feb 22; (36), 462-475. 
%\newline $\ddagger$ {\it These authors contributed equally to this work.}
%\begin{center} Featured as a {\it Science Magazine} ScienceShot and as a AAAS EurekAlert! \end{center}

\item Kevin M. Dean et al.  Deconvolution-Free Subcellular Imaging with Axially Swept Light Sheet Microscopy.  {\it Biophys. J.} 2015 Jun 16; 108(12), 2807-2815.
%\begin{center} Featured on the cover of {\it Biophysical Journal}, in {\it Nature Methods} 'News and Views.', and the Royal Microscopical Society's Magazine, {\it inFocus}.\end{center}

\item Kevin M. Dean et al,  High-Speed Multiparameter Photophysical Analyses  of Fluorophore Libraries. {\it Anal. Chem}.  2015 Apr 21; 87(10), 5026-30.

\item Kevin M. Dean et al, Microfluidics-Based Selection of Red-Fluorescent Proteins with Decreased Rates of Photobleaching.  {\it Int. Biol.}  2014 Nov 21; 7(2), 263-73.

\item Kevin M. Dean and Reto Fiolka.  Uniform and Scalable Light-Sheets Generated by Extended Focusing.  {\it Opt. Express}.  2014  Oct 16;22(21),26141-26152.
%\begin{center} {\it Featured in the Virtual Journal for Biomedical Optics (VJBO)}\end{center}

\item Kevin M. Dean and Amy E. Palmer.  Advances in Fluorescence Labeling Strategies for Dynamic Cellular Imaging.   {\it Nat. Chem. Biol}. 2014 May 16;10(7):512-23. 

\item Yan Qin et al, Direct Comparison of a Genetically Encoded Sensor and Small Molecule Indicator: Implications for Quantification of Cytosolic Zn\textsuperscript{2+}. {\it ACS Chem. Biol}. 2013 Aug 30;8(11):2366-71.

\item Lloyd M. Davis et al, Microfluidic Cell sorter for Use in Developing Red Fluorescent Proteins with Improved Photostability. {\it Lab Chip}. 2013 Jun 21;13(12):2320-7.

\item Kevin M. Dean, Yan Qin, and Amy E. Palmer.  Visualizing Metal Ions in Cells:  An Overview of Analytical Techniques, Approaches, and Probes. {\it Biochim. et Biophys. Acta.}  2012  Sep;1823(9):1406-15.

\item Jennifer L. Lubbeck et al, Microfluidic Flow Cytometer for Quantifying Photobleaching of Fluorescent Proteins in Cells. {\it Anal. Chem.} 2012 May 1;84(9):3929-37.

\item Kevin M. Dean et al, Analysis of Red-Fluorescent Proteins Provides Insight Into Dark-State Conversion and Photodegradation.  {\it Biophys. J}. 2011 Aug 17;101(4):961-9.

\item Kevin M. Dean and J. Charles Williamson. The Stir-Settle Approach to Semiautomated Light Scattering Determination of Liquid-Liquid Coexistence Curves.  {\it J. Chem. Eng. Data}.  2011 Jan 26;56(4):1433-7. 

\item Kevin. M. Dean et al, The Accuracy of Liquid-Liquid Phase Transition Temperatures Determined from Semiautomated Light Scattering Measurements.  {\it J. Chem. Phys}. 2010 Aug 21;133, 074506.

\end{etaremune}