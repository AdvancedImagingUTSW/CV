\documentclass[10pt]{res} 
\usepackage{helvet} 
\usepackage{enumitem}
\usepackage{etaremune}
\newsectionwidth{0pt} 
\setlength{\parindent}{0pt}
\newcommand{\forceindent}{\leavevmode{\parindent=2em\indent}}
\begin{document}

\name{Kevin M. Dean, Ph.D.} 
\address{{\bf Work Address}  \\ 6000 Harry Hines Blvd.  NL5.116A  \\ Dallas, TX 75390 \\ (214) 648-4833} 
\address{{\bf Permanent Address} \\ 4420 Hollow Oak Dr. \\ Dallas, TX 75287 \\ (303) 681-8863} 


\begin{resume}

\section{\centerline{VISION}}  \vspace{4pt} 
\forceindent As a scientist, I am driven to develop and apply cutting-edge imaging technologies that provide molecular insight into incredibly complex biological processes and systems.  My scientific training spans multiple disciplines, including biology, biochemistry, computer vision, fluorescence microscopy, and optical probes (including optogenetics, chemogenetics, and biosensors), and I have a strong history of leading and completing highly-technical, interdisciplinary, and collaborative research projects. 

\vspace{0.1in} \section{\centerline{RESEARCH AND WORK ACTIVITIES}}  \vspace{4pt} 

{\sl Tenure-Track Assistant Professor} \hfill 2022 -- Present \\ 
Lyda Hill Department of Bioinformatics and Cecil H. and Ida Green Center for Systems Biology. \newline UT Southwestern Medical Center, Dallas, TX, USA. \hfill \\
Advanced fluorescence imaging of tissues, single cell sequencing, and other single-cell approaches have provided incredible insight into the earliest events in the metastatic cascade, including the growth of the primary tumor and invasion of single cells or cell clusters into adjacent tissues. In contrast, the molecular and cellular processes that enable colonization of a remote tissue are poorly understood and data are scarce. The overarching goal of the Dean Lab is to identify the molecular mechanisms that enable cancer cells to colonize and populate a distant tissue. To achieve this, we are developing a series of self-driving microscopes that leverage cutting-edge computer vision routines and adaptively image diverse specimens, and advancing integrated multiplexing imaging approaches using a combination of hydrogel-based tissue clearing and directed evolution of labeling technologies. 
\newline

{\sl Research-Track Assistant Professor, and Director of the Microscopy Innovation Lab} \hfill 2018 -- 2022 \\ 
Department of Cell Biology. \newline UT Southwestern Medical Center, Dallas, TX, USA. \hfill \\
\forceindent The pace of microscopy technology development far outstrips the rate at which it can be commercialized and delivered to the greater biological community.  The Microscopy Innovation Lab at UT Southwestern works to eliminate this delay by developing potentially transformative instrumentation that is tailored to address specific biological questions, and engaging in in close collaborations with life scientists to gain molecular insight into physiological and pathophysiological phenomena. Towards these means, I work one-on-one with diverse biomedical scientists and physicians to develop custom imaging workflows that synergistically combine advances in optical probes, sample preparation, instrument design and operation, and computer vision-based analyses of TB-scale datasets. Imaging systems include 1) two cleared tissue Axially Swept Light-Sheet Microscopes that achieve 300 and 650 nm isotropic resolution, respectively, throughout centimeter-scale optically cleared tissues, 2) a high-speed single-objective oblique plane light-sheet microscope with 200 and 450 nm lateral and axial resolution, respectively, 3) a hyperspectral TIRF system capable of imaging 6 genetically encoded fluorescent probes in living cells, 4) a 2 and 3-photon laser scanning confocal equipped with adaptive optics, extended depth of focusing, and rapid axial scanning for ultradeep intravital imaging, and 5) a mesoSPIM microscope for macro-scale imaging with 5 $\mu$m isotropic resolution. Furthermore, through collaborations with the Danuser and Fiolka labs, a 6) 2-photon micro-environmental selective plane illumination microscope with 300 nm isotropic resolution, 7) a live-cell 1-photon Axially Swept Light-Sheet Microscope with 350 nm isotropic resolution, 7) a Field Synthesis variant of Lattice Light-Sheet Microscopy, 8) and a multi-directional illumination light-sheet microscope for imaging of developing organisms. Research combines biology, biophysics, optics, and computer vision.  
\newline

{\sl Founder} \hfill 2019 -- 2022 \\ 
Discovery Imaging Systems, LLC, Dallas, TX, USA. \hfill \\
https://www.discoveryimagingsystems.com \\
\forceindent Established an LLC to provide customized and world-class imaging solutions to academic research centers.  Business portfolio focused on new methods for imaging chemically transparent biological specimens with the cleared tissue variants of Axially Swept Light-Sheet Microscopy.  Customer purchases third-party components, and Discovery Imaging Systems provides expert assembly on site and annual maintenance thereafter.

{\sl Postdoctoral Fellow, Fiolka and Danuser Laboratories} \hfill 2014 -- 2018 \\ 
Lyda Hill Department of Bioinformatics \\
UT Southwestern Medical Center, Dallas, TX, USA. \hfill \\
\forceindent Development and application of light-sheet microscopy and computer vision technologies to elucidate mechanisms of cell-extracellular matrix adhesions, Rho GTPase signaling, and cytoskeletal dynamics of breast cancer cells disseminating into 3D tissue-like environments. Ruth L. Kirschstein Postdoctoral Fellow.  Research conducted under the guidance of Drs. Reto Fiolka and Gaudenz Danuser.

{\sl Director, BioFrontiers Advanced Imaging Resource} \hfill 2013 -- 2014 \\ 
BioFrontiers Institute \\
University of Colorado, Boulder, CO, USA. \hfill \\
\forceindent Established first campus-wide microscopy facility and expanded user base expanded to over 100 users from eight departments, two universities, and two biotechnology firms. Led a collaborative team in the building and development of novel microscopy and super-resolution instrumentation and techniques; executed independent research; provided facility strategic vision; organized analytical training for users; worked with researchers to develop customized protocols and assays; collaborated with faculty to compete for grants and contracts to support facility research activities and lead infrastructure grants;  interfaced and worked closely with industrial users.  Directly reported to the Chief Scientific Officer, Dr. Leslie Leinwand, and Dr. Tom Cech.

{\sl Graduate Research Assistant, Palmer Laboratory.} \hfill 2007 -- 2013 \\ 
Department of Biochemistry \\
University of Colorado, Boulder, CO, USA. \\
\forceindent Investigated the role of site-specific mutations in dark-state conversion and irreversible photobleaching for red-fluorescent proteins; developed high-throughput microfluidic cell-sorter capable of measuring fluorescence lifetime and the rate of photobleaching on single mammalian cells; directed-evolution of red-fluorescent proteins for improved photostability; collaborated with other lab members on the development and evaluation of optical biosensors.  Research conducted under the guidance of Drs. Amy E. Palmer and Ralph Jimenez. 

 {\sl Undergraduate Research Assistant, Williamson Laboratory.} \hfill 2003 -- 2006 \\ 
Department of Chemistry \\
Willamette University, Salem, OR, USA. \hfill \\
\forceindent Development of semi-automated laser light scattering instrumentation and the measurement of highly accurate and precise binary liquid-liquid phase diagrams.  Research conducted under the guidance of Dr. J. Charles Williamson.

\newpage
\section{\centerline{EDUCATION}} \vspace{4pt} 
{\sl Ph.D.}, Biochemistry \hfill GPA: 3.9 
\\ University of Colorado, Boulder, CO, USA. \hfill 2007 - 2013 \\ 
{\it Dissertation:} Fluorescent Proteins: Spectroscopic Studies, Microfluidic Analysis, and Generation of Improved Variants.  Research conducted under the guidance of Drs. Amy E. Palmer and Ralph Jimenez. \\

{\sl B.A.}, Chemistry \hfill GPA: 3.64 
\\ Willamette University, Salem, OR, USA. \hfill December 2002 - 2006\\
{\it Thesis:} Laser Light Scattering Investigations of the Isobutyric Acid/Water Binary Liquid System.  Research conducted under the guidance of Dr. J. Charles Williamson.


\section{\centerline{OTHER TRAINING}} \vspace{4pt} 
Marine Biological Laboratory, Woods Hole, MA. (2016)\\
Computational Image Analysis, Summer Course. \\
Directors: Lani Wu, Steven Altschuler, and Gaudenz Danuser


\vspace{0.1in} \section{\centerline{GRANT SUPPORT}} \vspace{12pt} 
\begin{enumerate}
[leftmargin=!,
labelindent=0pt,
itemindent=-66pt,
label=\textbullet]

\item 2023 -- 2024 
\hspace{8pt}
Harold C. Simmons Comprehensive Cancer Center, UTSW. \\
Next-Generation Histopathology Using Rapid Ultraviolet Photoacoustical Microscopy . \\
{\it Principal Investigator}


\item 2021 -- 2026 
\hspace{8pt}
1RM1GM145399 \\ 
National Institute of General Medical Sciences, NIH. \\
UTSW-UNC Center for Cell Signaling Analysis. \\
{\it Principal Investigator}

\item 2021 -- 2026 
\hspace{8pt}
1U54CA268072 \\ 
National Cancer Institute, NIH. \\
Intelligent Hyperspectral Imaging of Subcellular Molecular States at the Whole Organ Level. \\
{\it Principal Investigator}

\item 2020 -- 2025 
\hspace{8pt} 
1R01DK127589 \\
National Institute of Diabetes and Digest and Kidney Diseases, NIH. \\
Prune Belly Syndrome: Mechanisms of Filamin A Mutations. \\
{\it Co-Investigator} 

\item 2020 -- 2024
\hspace{8pt} 
1R01MH120131 \\ 
National Institute of Mental Health, NIH. \\
Neuronal Signaling Mechanisms of Stress-Induced Anhedonia in the Lateral Habenula. \\
{\it Collaborator}

\item 2019 -- 2020 
\hspace{8pt} 
1S10OD026722 \\
National Institute of General Medical Sciences, NIH. \\
Laser for Ultra-Deep Intravital Imaging.\\
{\it Key Personnel}

\item 2016 -- 2018 
\hspace{8pt} 
F32GM117793 \\
National Institute of General Medical Sciences, NIH. \\
Symmetry breaking and polarization of cells in 3D environments. \\
{\it Principal Investigator} 

\item 2011 -- 2013 
\hspace{8pt} 
0801680 \\
Division Of Graduate Education, NSF \\
Integrative Graduate Education and Research Trainee (IGERT) in Computational, Optical, Sensing and Imaging. \\
{\it Trainee}

\item 2009 -- 2011
\hspace{8pt}
T32GM065103 \\
National Institute of General Medical Sciences, NIH. \\
Interdisciplinary Predoctoral Training in Molecular Biophysics. \\
{\it Trainee}

\item 2006 -- 2007 
\hspace{8pt} 
Sigma Xi Undergraduate Research Fellowship. \\
{\it Trainee}

\end{enumerate}

\vspace{0.1in} 
\section{\centerline{PEER-REVIEWED PUBLICATIONS}} 
\vspace{12pt} 

\begin{etaremune}

\item Philippe Roudot et al., u-track 3D: measuring and interrogating intracellular dynamics in three dimensions.  {\it Submitted}.

\item Erik S. Welf et al., Worrying drives cell migration in mechanically unrestrained environments. {\it Submitted}.

\item Tadamoto Isogai, Kevin M. Dean et al., Direct Arp2/3-vinculin binding is essential for cell spreading, but only on compliant substrates and in 3D. {\it Submitted.}  

\item Andrew D. Weems et al., Blebs Promote Cell Survival by Assembling Oncogenic Signaling Hubs. {\it Accepted, Nature}.

\item Alec Bancroft et al., Discoidin Domain Receptor 2 regulates aberrant mesenchymal progenitor cell fate and matrix organization. {\it Accepted, Sci. Advances}.

\item Bingying Chen et al., Increasing the field-of-view in oblique plane microscopy via optical tiling. {\it Biomed. Opt. Exp.} 2022 Nov;(13) 5616-5627.

\item Bingying Chen et al., Resolution doubling in light-sheet microscopy via oblique plane structured illumination. {\it Nat. Methods} 2022 Sep;

\item Kevin M. Dean et al., Isotropic imaging across spatial scales with axially swept light-sheet microscopy. {\it Nat. Protoc.} 2022 Jul; (17) 2025-2053.
%\begin{center} Featured on the cover of {\it Nat. Protoc.}.\end{center}

\item Sue Y. Kim et al., Particle Retracking Algorithm Capable of Quantifying Large, Local Matrix Deformation for Traction Force Microscopy. {\it PLoS ONE.} 2022 Jun; (17) e0268614. 

\item Iliodora V Pop et al., Structure of long-range direct and indirect spinocerebellar pathways as well as local spinal circuits mediating proprioception. {\it Neursci.} 2022 Jan; (42) 581-600.

\item Hao Liu et al., Heterozygous mutation of Vegfr3 decreases renal lymphatics but is dispensable for renal function.{\it J. Am. Soc. Neph.} 2021 Sep; (32) 3099-3113.

\item Bo-Jui Chang et al., Real-time multi-angle projection imaging of biological dynamics. {\it Nat. Methods.} 2021 Jun; (18) 829-834.
%\begin{center} Featured in the NIH Director's Blog \end{center}

\item Sangyoon J. Han et al., Pre-complexation of talin and vinculin without tension is required for efficient nascent adhesion maturation. {\it eLife.} 2021 Mar; (10) e66151. {\it Co-corresponding author.}

\item Erik S. Welf et al., Actin-membrane release initiates cell protrusion.  {\it Dev. Cell}. 2020 Dec; (55) 723-736.

\item  Etai Sapoznik et al., A versatile Oblique Plane Microscope for large-scale and high-resolution imaging of subcellular dynamics.  {\it eLife.} 2020 Nov; (9) e57681. {\it Co-corresponding author.}
%\begin{center} Featured as a Method to Watch in {\it Nat. Methods} \end{center}

\item Kyung-min Lee et al., Proline rich 11 (PRR11) overexpression amplifies PI3K signaling and promotes antiestrogen resistance in breast cancer. {\it Nat. Comm}. 2020 Oct; (11) 1-15.

\item Tonmoy Chakraborty, Bingying Chen et al., Converting Lateral Scanning Into Axial Focusing to Speed Up 3D Microscopy. {\it Light Sci. Appl.} 2020 Sep; (9) 00401-00409.

\item Bo-Jui Chang, Kevin M. Dean, and Reto Fiolka.  Systematic and quantitative comparison of lattice and Gaussian light-sheets.  {\it Opt. Express.} 2020 Aug; (28) 27052-27077.
%\begin{center} Highlighted as an Editor's Pick in {\it Opt. Express} \end{center}

\item Bingying Chen, Tonmoy Chakraborty et al.,  Extended depth of focus multiphoton microscopy via incoherent pulse splitting. {\it Biomed. Opt. Express.} 2020 Jun; (11) 3830-3842.

\item Tonmoy Chakraborty et al., Light-sheet microscopy of Cleared Tissues with Isotropic, Subcellular Resolution. {\it Nat. Methods.} 2019 Nov; (16), 1109-1113.  {\it Co-corresponding author.}
%\begin{center} Technology Feature in {\it Nature} \end{center}

\item Meghan K. Driscoll et al., Robust and automated detection of subcellular morphological motifs in 3D microscopy images. {\it Nat. Methods.} 2019 Sept; (16) 1037-1044.  

\item Ashwathi S. Mohan et al., Enhanced dendritic actin network formation in lamellipodia drives proliferative signaling in growth-challenged Rac1\textsuperscript{P29S} melanoma cells.  
{\it Dev. Cell}.  2019 May; (49) 444-460.

\item Bo-Jui Chang, Mark Kittisopikul et al., Universal Light-Sheet Generation with Field Synthesis. {\it Nat. Methods}. 2019  Feb; (16) 235-238. 
%\begin{center}  Featured in {\it Nature Methods} "News \& Views." \end{center}

\item Kevin M. Dean and Reto Fiolka.  Lossless Three-Dimensional Parallelization in Digitally Scanned Light-Sheet Fluorescence Microscopy.  {\it Sci. Rep.} 2017 Aug; (7), 9332.  

\item Kevin M. Dean et al., Imaging Subcellular Dynamics with Fast and Light-Efficient Volumetrically Parallelized Microscopy. {\it Optica}.  2017 Feb; (4), 263-271.
%\begin{center} Featured on the cover of {\it Optica} and {\it Nature Methods} "News and Views." \end{center}

\item Jun Chu et al.,  A Bright Cyan-Excitable Orange Fluorescent Protein for Dual-Emission Microscopy and Highly Sensitive Bioluminescence Imaging In Vivo.  {\it Nat. Biotechnol}. 2016 May 30; (34), 760-767.

\item J. Charles Williamson et al.  Determination of Liquid-Liquid Critical Point Composition Using 90$^{\circ}$ Laser Light Scattering. {\it Phys. Rev. E.}  2016 Apr 21; (93), 042610.

\item Kevin M. Dean, Philippe Roudot et al,  Diagonally Scanned Light-Sheet Microscopy for Fast Volumetric Imaging of Adherent Cells.  {\it Biophys. J}.  2016 Mar 29; (110), 1456-1465. 
%$\newline 
%\ddagger$ {\it These authors contributed equally to this work.}
%\begin{center} Featured as an 'Emerging Biophysical Method' and the 'Best Of' reprint collection of {\it Biophysical Journal}. \end{center}

\item Erik S. Welf, Meghan K. Driscoll et al., Quantitative Multiscale Cell Imaging in Controlled 3D Microenvironments. {\it Dev. Cell}.  2016 Feb 22; (36), 462-475. 
%\newline $\ddagger$ {\it These authors contributed equally to this work.}
%\begin{center} Featured as a {\it Science Magazine} ScienceShot and as a AAAS EurekAlert! \end{center}

\item Kevin M. Dean et al.  Deconvolution-Free Subcellular Imaging with Axially Swept Light Sheet Microscopy.  {\it Biophys. J.} 2015 Jun 16; 108(12), 2807-2815.
%\begin{center} Featured on the cover of {\it Biophysical Journal}, in {\it Nature Methods} 'News and Views.', and the Royal Microscopical Society's Magazine, {\it inFocus}.\end{center}

\item Kevin M. Dean et al,  High-Speed Multiparameter Photophysical Analyses  of Fluorophore Libraries. {\it Anal. Chem}.  2015 Apr 21; 87(10), 5026-30.

\item Kevin M. Dean et al., Microfluidics-Based Selection of Red-Fluorescent Proteins with Decreased Rates of Photobleaching.  {\it Int. Biol.}  2014 Nov 21; 7(2), 263-73.

\item Kevin M. Dean and Reto Fiolka.  Uniform and Scalable Light-Sheets Generated by Extended Focusing.  {\it Opt. Express}.  2014  Oct 16;22(21),26141-26152.
%\begin{center} {\it Featured in the Virtual Journal for Biomedical Optics (VJBO)}\end{center}

\item Kevin M. Dean and Amy E. Palmer.  Advances in Fluorescence Labeling Strategies for Dynamic Cellular Imaging.   {\it Nat. Chem. Biol}. 2014 May 16;10(7):512-23. 

\item Yan Qin et al., Direct Comparison of a Genetically Encoded Sensor and Small Molecule Indicator: Implications for Quantification of Cytosolic Zn\textsuperscript{2+}. {\it ACS Chem. Biol}. 2013 Aug 30;8(11):2366-71.

\item Lloyd M. Davis et al., Microfluidic Cell sorter for Use in Developing Red Fluorescent Proteins with Improved Photostability. {\it Lab Chip}. 2013 Jun 21;13(12):2320-7.

\item Kevin M. Dean, Yan Qin, and Amy E. Palmer.  Visualizing Metal Ions in Cells:  An Overview of Analytical Techniques, Approaches, and Probes. {\it Biochim. et Biophys. Acta.}  2012  Sep;1823(9):1406-15.

\item Jennifer L. Lubbeck et al., Microfluidic Flow Cytometer for Quantifying Photobleaching of Fluorescent Proteins in Cells. {\it Anal. Chem.} 2012 May 1;84(9):3929-37.

\item Kevin M. Dean et al., Analysis of Red-Fluorescent Proteins Provides Insight Into Dark-State Conversion and Photodegradation.  {\it Biophys. J}. 2011 Aug 17;101(4):961-9.

\item Kevin M. Dean and J. Charles Williamson. The Stir-Settle Approach to Semiautomated Light Scattering Determination of Liquid-Liquid Coexistence Curves.  {\it J. Chem. Eng. Data}.  2011 Jan 26;56(4):1433-7. 

\item Kevin. M. Dean et al., The Accuracy of Liquid-Liquid Phase Transition Temperatures Determined from Semiautomated Light Scattering Measurements.  {\it J. Chem. Phys}. 2010 Aug 21;133, 074506.

\end{etaremune}


\vspace{0.1in}  \section{\centerline{PATENTS AND PATENT APPLICATIONS}} \vspace{12pt} 
\begin{enumerate}
\item 13,360,706 -- Optically Integrated Microfluidic Cytometer for High Throughput Screening of Photophysical Properties of Cells or Particles.  
\item 62,155,980 -- Uniform and Scalable Light-Sheets Generated by Extended Focusing.  
\item 16,093,561 -- Light-Sheet Microscope with Parallelized 3D Image Acquisition. 
\end{enumerate}

\vspace{0.1in} \section{\centerline{INVITED TALKS}} \vspace{12pt}  \begin{enumerate}
\item 2023 -  Willamette University.  Salem, OR, USA.
% \item { Kevin M. Dean}. Janelia-EMBL Bioimaging Seminar. Fall, 2022.
\item 2022 - Photonics Media Webinar Series.   Virtual.
\item 2022 - CLEO: Laser Science to Photonic Applications Technical Conference. San Jose, CA, USA.
\item 2022 - Light-Sheet Conference and Workshop, Marine Biological Laboratory.  Woods Hole, MS, USA.
\item 2021 - Institute for Quantitative Health Science \& Engineering, Michigan State University. Lansing, MI, USA.
\item 2020 - Photonics Media Webinar Series.  Virtual.
\item 2020 - Imaging ONE WORLD Series. Virtual.
\item 2019 - Next Generation Microscopy Workshop.  Cambridge, U.K.
\item 2017 - Cell Biology Seminar Series, Denver University, Denver, CO, USA.
\item 2014 - Andor Academy, University of Colorado, Boulder, CO, USA.
\item 2019 - American Chemical Society, Orlando, FL, USA. 
%\item Three-dimensional imaging with high-spatiotemporal resolution.  American Chemical Society, Orlando, FL.  March 31-April 4, 2019.    
\end{enumerate}


\vspace{0.1in} \section{\centerline{INVITED WORKSHOPS}} \vspace{12pt} 
\begin{enumerate}  
\item M2PC Workshop.  Pacific Northwest National Labs. Virtual. September 21-22, 2021.
\item Software for Microscopy.  Janelia Farm, Howard Hughes Medical Institute.  Ashburn, VA. February 19-21, 2020. 
\end{enumerate}


\vspace{0.1in} \section{\centerline{OTHER TALKS, PRESENTATIONS, AND CONFERENCE PAPERS}} \vspace{12pt} 
\begin{enumerate}
\item 2021 - Kevin M. Dean.  American Society for Cell Biology (ASCB). Virtual. 
\item 2021 - Kevin M. Dean.  Society for Neuroscience (SfN). Virtual. 
\item 2021 - Kevin M. Dean.  EMBO \& EMBL Symposium: Seeing is Believing - Imaging the Processes of Life.  Virtual.  
\item 2021 - Doug P. Shepherd, Alfred Millett-Sikking, Andrew G. York, Reto P. Fiolka, Kevin M. Dean.  Focus On Microscopy. Virtual.
\item 2021 - Kevin M. Dean, Denise Ramirez, Reto Fiolka, Hu Zhao, and Kate Luby-Phelps. Arizona Imaging and Microanalysis Conference. Virtual. 
\item 2020 - Tonmoy Chakraborty, Reto Fiolka, and Kevin M. Dean.  The Network of European Bioimage Analysts (NEUBIAS).  Bordeaux, France.
\item 2020 - Kevin M. Dean.  Imaging Mouse Development.  Janelia Farm, Howard Hughes Medical Institute.  Ashburn, VA, USA. 
\item 2020 - Alfred Millett-Sikking, Andrew G. York, Reto Fiolka, Kevin M. Dean.  Focus On Microscopy.  Osaka, Japan. {\it Cancelled}.
\item 2020 - Kevin M. Dean.  The Academy of Medicine, Engineering \& Science of Texas (TAMEST).  Dallas, TX, USA. 
\item 2019 - Tonmoy Chakraborty, Reto Fiolka, and Kevin M. Dean.  EMBL Seeing is Believing. Heidelberg, Germany. 
\item 2019 - Kevin M. Dean. Frontiers and Challenges in Laser-Based Biological Microscopy. Telluride, CO, USA.
\item 2019 - Tonmoy Chakraborty, Bo-Jui Chang, Kevin M. Dean, and Reto Fiolka.  Focus On Microscopy.  London, U.K.  
\item 2018 - Kevin M. Dean, Tadamoto Isogai, Philippe Roudot, Erik S. Welf, Meghan K. Driscoll, Reto Fiolka, and Gaudenz Danuser.  Focus On Microscopy.  Singapore, Republic of Singapore. 
\item 2017 - Kevin M. Dean, Erik S. Welf, Philippe Roudot, Meghan K. Driscoll, Reto Fiolka, and Gaudenz Danuser.  Frontiers in Imaging Science.  Janelia Farm, Howard Hughes Medical Institute.  Ashburn, VA, USA.  
\item 2017 - Kevin M. Dean, Erik S. Welf, Philippe Roudot, Meghan K. Driscoll, Reto Fiolka, and Gaudenz Danuser.  Biophysical Society.  New Orleans, LA, USA. 
\item 2017 - Kevin M. Dean, Erik S. Welf, Philippe Roudot, Meghan K. Driscoll, Reto Fiolka, and Gaudenz Danuser.  Quantitative BioImaging 2017.  College Station, TX, USA. 
\item 2016 - Kevin M. Dean et al.,  Microscope Technologies for Quantitative Imaging in 3D Microenvironments.  {\it SPIE, Biomedical Optics and Medical Imaging}.  
\item 2016 - Kevin M. Dean and Reto Fiolka. Isotropic 3D Imaging over Large Volumes with Axially Swept Light-Sheet Microscopy.  {\it inFocus Magazine, Royal Microscopical Society}.  
\item 2016 - Kevin M. Dean, Erik S. Welf, Philippe Roudot, Meghan K. Driscoll, Reto Fiolka, and Gaudenz Danuser.  Gordon Research Conference - Signaling by Adhesion Receptors.  Bates College, Lewiston, Maine, USA. 
\item 2015 - Kevin M. Dean, Philippe Roudot, Erik S. Welf, Meghan K. Driscoll, Reto Fiolka, and Gaudenz Danuser.  EMBO \& EMBL Symposium: Seeing is Believing - Imaging the Processes of Life.  EMBL, Heidelberg, Germany.  
\item 2013 - Kevin M. Dean, Jennifer L. Lubbeck, Premashis Manna, Lloyd M. Davis, Ralph Jimenez, Amy E. Palmer.  EMBO \& EMBL Symposium: Seeing is Believing - Imaging the Processes of Life.  EMBL, Heidelberg, Germany.
\item 2013 - Kevin M. Dean, Jennifer L. Lubbeck, Lloyd M. Davis, Ralph Jimenez, Amy E. Palmer.  Signaling and Cellular Regulation Symposium.  University of Colorado, Boulder, CO, USA.  
\item 2013 - Kevin M. Dean, Jennifer L. Lubbeck, Lloyd M. Davis, Ralph Jimenez, Amy E. Palmer.  Computational, Optical, Sensing and Imaging.  University of Colorado, Boulder, CO, USA.  
\item 2012 - Kevin M. Dean, Jennifer L. Lubbeck, Lloyd M. Davis, Ralph Jimenez, Amy E. Palmer.  Computational Optical, Sensing and Imaging IAB Meeting.  Breckenridge, CO, USA.  
\item 2011 - Kevin M. Dean, Jennifer L. Lubbeck, Lloyd M. Davis, Ralph Jimenez, Amy E. Palmer.  Colorado Photonics Industry Association Annual Meeting.  Boulder, CO, USA.    
\item 2011 - Kevin M. Dean, Jennifer L. Lubbeck, Lloyd M. Davis, Ralph Jimenez, Amy E. Palmer.  Molecular and Cellular Biophysics Symposium at the University of Denver.  Denver, CO, USA. 
\item 2011 - Kevin M. Dean, Jennifer L. Lubbeck, Lloyd M. Davis, Ralph Jimenez, Amy E. Palmer.  Multiphoton Imaging: The Next 6$\times$10\textsuperscript{23} Femtoseconds, Janelia Farm, Howard Hughes Medical Institute.  Ashburn, VA, USA. 
\item 2010 - Kevin M. Dean, Jennifer L. Lubbeck, Hairong Ma, Ralph Jimenez, Amy E. Palmer.  New Optical Methods in Cell Physiology, Society of General Physiologists. Wood's Hole, MA, USA. 
\item 2010 - Kevin M. Dean, Jennifer L. Lubbeck, Hairong Ma, Ralph Jimenez, Amy E. Palmer.  Novel Approaches to Bioimaging II, Janelia Farm, Howard Hughes Medical Institute.  Ashburn, VA, USA.
\item 2006 - Kevin M. Dean, Joseph C. Williamson, Student Scholarship Recognition Day, Salem, OR, USA.
\item 2006 - Kevin M. Dean, Joseph C. Williamson, Northwest Undergraduate Science Research Conference at Oregon Health \& Science University, Portland, OR, USA.
\item 2005 - Kevin M. Dean, Joseph C. Williamson, Student Scholarship Recognition Day, Salem, OR, USA.
\end{enumerate}


\vspace{0.1in} \section{\centerline{CONFERENCE ORGANIZING}} \vspace{12pt} 
\begin{enumerate}
\item CLEO: Laser Science to Photonic Applications Technical Conference. 
\item Biomedical Applications Program Committee. OSA Biophotonics Congress: Optics in the Life Sciences.
\item BRAIN Technical Program Committee. OSA Biophotonics Congress: Optics in the Life Sciences.
\end{enumerate}


\vspace{0.1in} \section{\centerline{GRANT REVIEW}} \vspace{2pt} 
2022 \hspace{58pt} Netherlands Organization for Scientific Research (NWO, the Dutch Research Council). \\
2019 \hspace{58pt} Netherlands Organization for Scientific Research (NWO, the Dutch Research Council). \\


\vspace{0.1in} \section{\centerline{REFEREE FOR JOURNALS}} \vspace{12pt} 
\begin{itemize}
\item Nature Methods
\item Nature Protocols
\item Nature Biotechnology
\item Scientific Reports
\item Journal of Neuroscience Methods
\item Optics Express
\item Biomedical Optics Express
\item Journal of Biophotonics
\item Biophysical Journal
\item PLoS One
\item EMBO Molecular Systems Biology 
\item eLife
\end{itemize}


\vspace{0.1in} \section{\centerline{TEACHING EXPERIENCE}} \vspace{2pt} 
2022 -- Present \hspace{14pt} Discussion Leader, Responsible Conduct of Research, UT Southwestern Medical Center. \\ 
2021 -- Present \hspace{14pt} Assistant Instructor, Computational Image Analysis, UT Southwestern Medical Center. \\
2015 -- 2019 \hspace{28pt} Assistant Instructor, Optical Microscopy, UT Southwestern Medical Center. \\
2013 -- 2014 \hspace{28pt} Laboratory Assistant, Light Microscopy, University of Colorado. \\
2007 -- 2008 \hspace{28pt} Teaching Assistant, Introductory Chemistry, University of Colorado.  \\
2006 -- 2007 \hspace{28pt} Tutor, Introductory Chemistry and Physical Chemistry, Willamette University. \vspace{6pt}

\vspace{0.1in} \section{\centerline{INSTITUTIONAL COMMITTEES}} \vspace{2pt} 
2022 -- Present \hspace{14pt} UT Southwestern Committee on Improving Research Staff Recruitment and Hiring. \\
2022 -- Present \hspace{14pt} Lyda Hill Dept. of Bioinformatics and BME-Computational Biology Retreat. \\
2021 -- Present \hspace{14pt} Dept. of Biophysics and Lyda Hill Dept. of Bioinformatics Faculty Search. \\
2021 -- Present \hspace{14pt}  Faculty Senate at UT Southwestern.  \\
2021 -- Present \hspace{14pt}  Medical School Admissions Committee.

\vspace{0.1in}
\section{\centerline{SCIENTIFIC MENTORSHIP}} \vspace{2pt} 
2022 -- Present \hspace{14pt} Zach Marin, Ph.D., Postdoctoral Fellow, UT Southwestern Medical Center. \\
2021 -- Present \hspace{14pt} Dax Collision, Undergraduate Student Intern, UT Southwestern Medical Center. \\
2021 -- Present \hspace{14pt} Jinlong Lin, Research Technician I, UT Southwestern Medical Center. \\
2021 -- Present \hspace{14pt} Hazel Borges, M.S., Research Assistant I, UT Southwestern Medical Center. \\
2021 -- Present \hspace{14pt} Renil Gupta, High-School Student Intern, UT Southwestern Medical Center. \\
2021 -- 2022 \hspace{28pt} Samir Mamtani, Undergraduate Student, University of Wisconsin. \\
2021 -- 2022 \hspace{28pt} Sampath Rapuri, Undergraduate Student, Johns Hopkins University. \\
2019 -- 2021 \hspace{28pt} Evgenia Azarova, Graduate Student, Johns Hopkins University. \\
2019 -- 2019 \hspace{28pt} Ryan Hammond, Undergraduate Student, Carnegie Mellon University. \\
2018 -- 2019 \hspace{28pt} Qiuyan Shao, Principal Data Scientist at Capital One. \\
2018 -- 2020 \hspace{28pt} Saumya Vora, M.S., Regulatory Affairs Specialist at Medtronic. \\
2012 -- 2014 \hspace{28pt} Pia Friis,  Laboratory Technician in Clinical Microbiomics. \\
2012 -- 2013 \hspace{28pt} Jordan Gringauz, Law Student, Washington College of Law. \\
2011 -- 2013 \hspace{28pt} Derek Gann, Owner and Engineer of CAFE Circuits and Software. \\
2010 -- 2011 \hspace{28pt} Jennifer Binder, Graduate Student, Arizona State University.

\vspace{0.1in}
\section{\centerline{DISSERTATION COMMITTEES}}  \vspace{2pt} 
2020 -- Present \hspace{14pt} Byron Weiss, Danuser Lab, Lyda Hill Dept. of Bioinformatics, UTSW. \\
2020 -- Present \hspace{14pt} Harshida Pancholi, Roberts Lab, Dept. of Neuroscience, UTSW. \\

\vspace{0.1in}
\section{\centerline{DISTINCTIONS}} 
\vspace{4pt} 
2022 \hspace{58pt} Inductee, Weed High School Hall of Fame. \\
2018 \hspace{58pt} Dean's Discretionary Award, UT Southwestern Medical Center.  \\
2018 \hspace{58pt} Runner-Up, UTSW Brown-Goldstein Excellence in Postdoctoral Research. \\
2016 \hspace{58pt} Nominee, AAAS/Science Program for Excellence in Science.  \\
2015 \hspace{58pt} Finalist, 2015 Educational Award, Edmund Optics. \\
2011 \hspace{58pt} Certificate of Achievement, Colorado Photonics Industry Association (CPIA). \\
2007 \hspace{58pt} Excellence in Graduate Teaching. \\
2007 \hspace{58pt} ESPN \& CoSIDA District 8 Academic All-American. \\
2007 \hspace{58pt} Inducted into National Football Foundation Hampshire Honor Society.  \\
2007 \hspace{58pt} Co-Recipient, Henry Booth Outstanding Senior Male Athlete Award. \\
2006 \hspace{58pt} Willamette University Bill Trenbeath Award. \\
2006 \hspace{58pt} ESPN \& CoSIDA District 8 Academic All-American. \\
2006 \hspace{58pt} Sigma Alpha Epsilon National Scholar. \\
2006 \hspace{58pt} Willamette University Athlete of The Month.\\
2006 \hspace{58pt} Northwest Conference Athlete of the Week. \\
2006 \hspace{58pt} Willamette University Football Team Captain. \\
2005 \hspace{58pt} Invited to Speak at the Murdock's NW Undergraduate Research Meeting. \\
2005 \hspace{58pt} Willamette University Football Team Captain. \\
2005 \hspace{58pt} Peterson Family Scholarship for Chemistry. \\
2005 \hspace{58pt} Inducted into the Mortar Board National Honor Society.  \\
2004 \hspace{58pt} Sigma Alpha Epsilon National Scholar.  \\
2004 \hspace{58pt} Inducted into the National Society of Collegiate Scholars (NSCS).  \\
2004 \hspace{58pt} Selected for Science Collaborative Research Program. \\
2002 \hspace{58pt} Haynes McHale Award.  \\
2002 \hspace{58pt} Neal Wade Award - Weed High School Outstanding Male Athlete of the Year. \\
2002 \hspace{58pt} Inducted into California Scholarship Federation.

\vspace{0.1in} 
\section{\centerline{LEADERSHIP POSITIONS}}
\vspace{2pt} 
2007 -- 2013 \hspace{26pt} Sigma Alpha Epsilon \-- University of Colorado Advisor and Regional Director \\
2007 \hspace{58pt} Coast 2 Coast 4 Kalan Memorial Bike Ride \-- Philanthropy Coordinator and Participant \\
2005 -- 2006 \hspace{26pt} Willamette University Football Team Captain

\vspace{0.1in} \section{\centerline{PHILANTHROPY \& COMMUNITY SERVICE}} \vspace{2pt}
2016 -- 2022 \hspace{26pt} Neonatal Support Staff, Parkland Neonatal Intensive Care Unit \\
2015 -- 2020 \hspace{26pt} Science Advocate, Science Policy, Education, and Communication club (SPEaC) \\
2006 \hspace{58pt} Participant and Organizer, Coast to Coast for Kalan Memorial Bicycle Ride \\
2004 -- 2005 \hspace{26pt} Participant, United Way Day of Caring \\
2004 -- 2005 \hspace{26pt} Participant, Willamette University Bearcat Day Student Athlete Youth Outreach \\
2002 \hspace{58pt} Participant, Habitat for Humanity \\

\vspace{0.1in} \section{\centerline{PROFESSIONAL MEMBERSHIPS}} \vspace{-6pt} 
\begin{center}
American Society for Cell Biology \\
American Chemical Society \\
Biophysical Society \\
Society for Neuroscience \\

\end{center}

\vspace{0.1in} \section{\centerline{INSTITUTIONAL TRAINING}}  \vspace{-6pt}
\begin{center}
Business Finance for Basic Science \\
Biological Safety\\
Computational Image Analysis in Cellular and Developmental Biology \\
Equal Employment Opportunity  \\
Fiscal Ethics and Procurement \\
Hazardous Materials and Waste Management \\
Human Trafficking \\
Information Security \\
Institutional Animal Care and Use \\
Occupational Health and Safety\\ 
Performance Evaluation for Faculty Managers \\
Phishing Awareness \\
Protecting Health Information \\
Responsible Conduct of Research \\
Title IX \\
Training for Intervention Procedures (TIPS)  \\
Texas Public Information Act (TPIA) \\
Youth Protection \\
\end{center}

%\pagebreak
%\vspace{0.1in}
%\section{\centerline{REFERENCES}}
%\center{

%\textbf{Dr. Gaudenz Danuser} \\(214) 648-3835\\gaudenz.danuser@utsouthwestern.edu \\UT Southwestern Medical Center \\6000 Harry Hines Blvd., NL5.120A \\Dallas, TX 75208\linebreak

%\textbf{Dr. Reto Fiolka} \\ (214) 648-4596 \\ reto.fiolka@utsouthwestern.edu \\ UT Southwestern Medical Center\\6000 Harry Hines Blvd., NL5.120A \\Dallas, TX 75208\linebreak

%\textbf{Dr. Amy E. Palmer} \\(303) 492-1945\\amy.palmer@colorado.edu\\University of Colorado Boulder\\3415 Colorado Ave.\\Boulder, CO 80303 \linebreak

%\textbf{Dr. Ralph Jimenez} \\(303) 492-8439\\rjimenez@jilau1.colorado.edu\\University of Colorado Boulder\\JILA, 440 UCB\\Boulder, CO 80309\linebreak

%\textbf{Dr. Tom Perkins} \\(303) 492-5291\\tperkins@jila.colorado.edu\\University of Colorado Boulder\\JILA, 440 UCB\\Boulder, CO 80309\linebreak

%\textbf{Dr. Leslie Leinwand} \\(303) 492-7606\\leslie.leinwand@colorado.edu\\University of Colorado Boulder\\3415 Colorado Ave.\\Boulder, CO 80309\\
%}
\end{resume} 
\end{document}