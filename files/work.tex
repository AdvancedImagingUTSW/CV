{\sl Tenure-Track Assistant Professor} \hfill 2022 -- Present \\ 
Lyda Hill Department of Bioinformatics and Cecil H. and Ida Green Center for Systems Biology, UT Southwestern Medical Center, Dallas, TX, USA. \hfill \\
\forceindent My lab aims to establish a broadly generalizable method for studying cell biological processes within complex tissues such as the hematopoietic niche. This includes creating advanced tissue processing techniques that enable interrogation of the location and activation state for ~20 different proteins with ~80 nm resolution, deep tissue imaging microscopes that adapt to and optimally probe the biological specimen, and cutting-edge software that comprehensively analyzes the architecture of single cells throughout entire tissues. Ultimately, our goal is to bridge the gap between traditional cell biology and the complex reality of living organisms, providing deeper insights into health and disease. Member of the Biomedical Engineering and Molecular Biophysics Graduate Programs, and the Simmons Comprehensive Cancer Center's Development and Cancer Program.

{\sl Executive Director, Cancer Cell Imaging Core (CCIC)} \hfill 2025 -- Present \\ 
UT Southwestern Medical Center, Dallas, TX, USA. \hfill \\
\forceindent Founded and established the Cancer Cell Imaging Core (CCIC), a centralized facility dedicated to democratizing access to advanced imaging technologies for the cancer research community across Texas. As Executive Director, I oversee a state-of-the-art platform that integrates custom light-sheet microscopy, expansion microscopy, cyclic multiplexed imaging, and automated tissue clearing to enable high-throughput, subcellular-resolution imaging of thick tissue specimens. The CCIC supports user training, collaborative research, and fee-for-service imaging workflows, providing optimized sample preparation pipelines, robotics-enabled imaging automation, and cloud-based image analysis. Under my direction, the CCIC builds upon technologies developed through multiple NIH- and CPRIT-funded initiatives, offering unprecedented capabilities for visualizing and quantifying protein localization, abundance, and activation states in situ. The CCIC also partners with other spatial biology cores to integrate functional imaging with transcriptomic and metabolic profiling and serves as a hub for statewide dissemination of cutting-edge imaging approaches.

{\sl Research-Track Assistant Professor, and Director of the Microscopy Innovation Lab} \hfill 2018 -- 2022 \\ 
Department of Cell Biology, UT Southwestern Medical Center, Dallas, TX, USA. \hfill \\
\forceindent The pace of microscopy technology development far outstrips the rate at which it can be commercialized and delivered to the greater biological community.  The Microscopy Innovation Lab at UT Southwestern works to eliminate this delay by developing potentially transformative instrumentation that is tailored to address specific biological questions, and engaging in in close collaborations with life scientists to gain molecular insight into physiological and pathophysiological phenomena. Towards these means, I work one-on-one with diverse biomedical scientists and physicians to develop custom imaging workflows that synergistically combine advances in optical probes, sample preparation, instrument design and operation, and computer vision-based analyses of TB-scale datasets. 

{\sl Founder} \hfill 2019 -- 2022 \\ 
Discovery Imaging Systems, LLC, Dallas, TX, USA. \hfill \\
%https://www.discoveryimagingsystems.com \\
\forceindent Established an LLC to provide customized and world-class imaging solutions to academic research centers.  Business portfolio focused on new methods for imaging chemically transparent biological specimens with the cleared tissue variants of Axially Swept Light-Sheet Microscopy.  Customer purchases third-party components, and Discovery Imaging Systems provides expert assembly on site and annual maintenance thereafter.

{\sl Postdoctoral Fellow, Fiolka and Danuser Laboratories} \hfill 2014 -- 2018 \\ 
Lyda Hill Department of Bioinformatics, UT Southwestern Medical Center, Dallas, TX, USA. \hfill \\
\forceindent Development and application of light-sheet microscopy and computer vision technologies to elucidate mechanisms of cell-extracellular matrix adhesions, Rho GTPase signaling, and cytoskeletal dynamics of breast cancer cells disseminating into 3D tissue-like environments. Ruth L. Kirschstein Postdoctoral Fellow.  Research conducted under the guidance of Drs. Reto Fiolka and Gaudenz Danuser.

{\sl Director, BioFrontiers Advanced Imaging Resource} \hfill 2013 -- 2014 \\ 
BioFrontiers Institute, University of Colorado, Boulder, CO, USA. \hfill \\
\forceindent Established first campus-wide microscopy facility and expanded user base expanded to over 100 users from eight departments, two universities, and two biotechnology firms. Led a collaborative team in the building and development of novel microscopy and super-resolution instrumentation and techniques; executed independent research; provided facility strategic vision; organized analytical training for users; worked with researchers to develop customized protocols and assays; collaborated with faculty to compete for grants and contracts to support facility research activities and lead infrastructure grants;  interfaced and worked closely with industrial users.  Directly reported to the Chief Scientific Officer, Dr. Leslie Leinwand, and Dr. Tom Cech.

{\sl Graduate Research Assistant, Palmer Laboratory.} \hfill 2007 -- 2013 \\ 
Department of Biochemistry, University of Colorado, Boulder, CO, USA. \\
\forceindent Investigated the role of site-specific mutations in dark-state conversion and irreversible photobleaching for red-fluorescent proteins; developed high-throughput microfluidic cell-sorter capable of measuring fluorescence lifetime and the rate of photobleaching on single mammalian cells; directed-evolution of red-fluorescent proteins for improved photostability; collaborated with other lab members on the development and evaluation of optical biosensors.  Research conducted under the guidance of Drs. Amy E. Palmer and Ralph Jimenez. 

 {\sl Undergraduate Research Assistant, Williamson Laboratory.} \hfill 2003 -- 2006 \\ 
Department of Chemistry, Willamette University, Salem, OR, USA. \hfill \\
\forceindent Development of semi-automated laser light scattering instrumentation and the measurement of highly accurate and precise binary liquid-liquid phase diagrams.  Research conducted under the guidance of Dr. J. Charles Williamson.